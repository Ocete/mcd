% Created 2021-10-06 mar 13:05
% Intended LaTeX compiler: pdflatex
%%% Local Variables:
%%% LaTeX-command: "pdflatex --shell-escape"
%%% End:
\documentclass[11pt]{article}
\usepackage[utf8]{inputenc}
\usepackage[T1]{fontenc}
\usepackage{graphicx}
\usepackage{grffile}
\usepackage{longtable}
\usepackage{wrapfig}
\usepackage{rotating}
\usepackage[normalem]{ulem}
\usepackage{amsmath}
\usepackage{textcomp}
\usepackage{amssymb}
\usepackage{listings}
\usepackage{capt-of}
\usepackage{hyperref}
\hypersetup{colorlinks=true, linkcolor=black}
\setlength{\parindent}{0in}
\usepackage[margin=1.1in]{geometry}
\usepackage[spanish]{babel}
\usepackage{mathtools}
\usepackage{palatino}
\usepackage{fancyhdr}
\usepackage{sectsty}
\usepackage{engord}
\usepackage{cite}
\usepackage{graphicx}
\usepackage{setspace}
\usepackage[compact]{titlesec}
\usepackage[center]{caption}
\usepackage{placeins}
\usepackage{tikz}
\usetikzlibrary{positioning}
\usetikzlibrary{bayesnet}
\usetikzlibrary{shapes.geometric}
\usetikzlibrary{decorations.text}
\usepackage{color}
\usepackage{amsmath}
\usepackage{minted}
\usepackage{pdfpages}
\usepackage[framemethod=TikZ]{mdframed} % Para entorno enunciado
\usepackage{amsthm}

\def\inline{\lstinline[basicstyle=\ttfamily,keywordstyle={}]}
\titlespacing*{\subsection}{0pt}{5.5ex}{3.3ex}
\titlespacing*{\section}{0pt}{5.5ex}{1ex}
\author{José Antonio Álvarez Ocete}
\date{\today}

% Comandos propios

\definecolor{myOrange}{rgb}{1, 0.733, 0.612}
\newcommand{\I}{\mathbb{I}}
\newcommand{\la}{\langle}
\newcommand{\ra}{\rangle}
\newcommand{\myparagraph}[1]{\paragraph*{ \\ #1}\mbox{}\\}
\newcommand{\R}{\mathbb{R}}
\newcommand{\E}{\mathbb{E}}
\newcommand{\Var}{\text{Var}}
%\newcommand{\log}{\text{log}}
% Para poner sonrisa sobre puntos suspensivos. Uso: \overplace{n}{\dotsc}
\newcommand{\overplace}[2]{%
	\overset{\substack{#1\\\smile}}{#2}%
}

\theoremstyle{plain}
\newtheorem*{theorem*}{Teorema}
\newtheorem{theorem}{Teorema}


\usepackage{mathtools}
\usepackage[shortlabels]{enumitem}
\DeclarePairedDelimiter\abs{\lvert}{\rvert}%
\usepackage{cancel}

%----------------------------------------------------------------------------------------
%	PACKAGES AND OTHER DOCUMENT CONFIGURATIONS
%----------------------------------------------------------------------------------------

\usepackage{blindtext} % Package to generate dummy text
\usepackage{charter} % Use the Charter font
\usepackage[utf8]{inputenc} % Use UTF-8 encoding
\usepackage{microtype} % Slightly tweak font spacing for aesthetics
\usepackage[english]{babel} % Language hyphenation and typographical rules
\usepackage{amsthm, amsmath, amssymb} % Mathematical typesetting
\usepackage{float} % Improved interface for floating objects
\usepackage[final, colorlinks = true, 
linkcolor = black, 
citecolor = black]{hyperref} % For hyperlinks in the PDF
\usepackage{graphicx, multicol} % Enhanced support for graphics
\usepackage{xcolor} % Driver-independent color extensions
\usepackage{marvosym, wasysym} % More symbols
\usepackage{rotating} % Rotation tools
\usepackage{censor} % Facilities for controlling restricted text
\usepackage{listings} % Environment for non-formatted code, !uses style file!
\usepackage{pseudocode} % Environment for specifying algorithms in a natural way
% Environment for f-structures, !uses style file!
\usepackage{booktabs} % Enhances quality of tables
\usepackage{tikz-qtree} % Easy tree drawing tool
% Configuration for b-trees and b+-trees, !uses style file!
%\usepackage[backend=biber,style=numeric,
%            sorting=nyt]{biblatex} % Complete reimplementation of bibliographic facilities
%\addbibresource{ecl.bib}
\usepackage{csquotes} % Context sensitive quotation facilities
\usepackage[yyyymmdd]{datetime} % Uses YEAR-MONTH-DAY format for dates
\renewcommand{\dateseparator}{-} % Sets dateseparator to '-'
\usepackage{fancyhdr} % Headers and footers
\pagestyle{fancy} % All pages have headers and footers
\fancyhead{}\renewcommand{\headrulewidth}{0pt} % Blank out the default header
\fancyfoot[L]{} % Custom footer text
\fancyfoot[C]{} % Custom footer text
\fancyfoot[R]{\thepage} % Custom footer text
\newcommand{\note}[1]{\marginpar{\scriptsize \textcolor{red}{#1}}} % Enables comments in red on margin

%-------------------------------
%	ENTORNO PARA ENUNCIADOS
%-------------------------------

%https://texblog.org/2015/09/30/fancy-boxes-for-theorem-lemma-and-proof-with-mdframed/
%\newcounter{enunciado}[section] - para resetear el counter por secciones
\newcounter{enunciado}
\setcounter{enunciado}{0}

% Esta línea pone la numeración de los ejercicios como {seccion}.{ejercicio}
%\renewcommand{\thetheo}{\arabic{section}.\arabic{enunciado}}
\newcommand{\thetheo}{\arabic{enunciado}}

\newenvironment{enunciado}[2][]{%
		\refstepcounter{enunciado}%
		\ifstrempty{#1}%
		{\mdfsetup{%
				frametitle={%
					\tikz[rounded corners, baseline=(current bounding box.east),outer sep=0pt]
					\node[anchor=east,rectangle,fill=myOrange]
					{\strut Ejercicio~\thetheo};}}
		}%
		{\mdfsetup{%
				frametitle={%
					\tikz[rounded corners, baseline=(current bounding box.east),outer sep=0pt]
					\node[anchor=east,rectangle,fill=myOrange]
					{\strut Ejercicio~\thetheo:~#1};}}%
		}%
		\mdfsetup{roundcorner=5pt, innertopmargin=10pt,linecolor=myOrange,%
			linewidth=2pt,topline=true,%
			frametitleaboveskip=\dimexpr-\ht\strutbox\relax
		}
		\begin{mdframed}[]\relax%
			%\label{#2}
	}
	{\end{mdframed}}

%-------------------------------
%	TITLE SECTION
%-------------------------------

\addtolength{\hoffset}{-2cm}
\addtolength{\textwidth}{4.1cm}
\addtolength{\voffset}{-2.25cm}
\addtolength{\textheight}{3cm}
\setlength{\parskip}{0pt}
\setlength{\parindent}{0in}

\begin{document}

\fancyhead[C]{}
\hrule \medskip % Upper rule
\begin{minipage}{0.295\textwidth} 
	\raggedright
	\footnotesize
	José Antonio Álvarez Ocete \hfill\\   
	77553417Q \hfill\\
	joseantonio.alvarezo@estudiante.uam.es
\end{minipage}
\begin{minipage}{0.4\textwidth} 
	\centering 
	\large 
	Entrega de problemas\\ 
	\normalsize 
	Teoría de la Información\\ 
\end{minipage}
\begin{minipage}{0.295\textwidth} 
	\raggedleft
	\today\hfill\\
\end{minipage}
\medskip\hrule 
\bigskip

%-------------------------------
%	CONTENTS
%-------------------------------

\begin{enunciado}
	
	Calcula la entropía de una variable $X \sim \text{N}(\mu, \sigma^2)$.
	
\end{enunciado}

Dada una variable aleatoria $X \sim \text{N}(\mu, \sigma^2)$, utilizamos la fórmula de la entropía para una variable contínua:

\[
	\begin{align*}
		H(X) & = - \E[log(X)] = - \int_\R f(x) \cdot \log_2 \bigg( \frac{1}{\sigma \sqrt{2\pi}} e^{\frac{-(x-\mu)^2}{2\sigma^ 2}} \bigg) dx \\
		& = - \int_\R f(x) \bigg( \log_2 \frac{1}{\sigma \sqrt{2\pi}} + \frac{-(x-\mu)^2}{2\sigma^ 2} \log_2 e \bigg) dx \\
		& = - \log_2 \frac{1}{\sigma \sqrt{2\pi}} \underbrace{\int_\R f(x) dx}_{= 1} + \frac{\log_2 e}{2\sigma^ 2} \underbrace{\int_\R f(x) (x-\mu)^2 dx}_{= \sigma^2} \\
		& = - \log_2 \frac{1}{\sigma \sqrt{2\pi}} + \frac{1}{2}\log_2 e \\
		& = \log_2 \sigma \sqrt{2\pi} + \frac{1}{2}\log_2 e \\
		& = \frac{1}{2} \log_2 2\pi\sigma^2  + \frac{1}{2}\log_2 e \\
		& = \frac{1}{2} \log_2 (2\pi\sigma^2 e) \\
	\end{align*}
\]
\qed

\begin{enunciado}
	
	Prueba que la información mútua entre dos variables aleatorias es simétrica. Esto es, $MI(X, Y) = MI(Y, X)$ para cualesquiera variables aleatorias $X,Y$.
	
\end{enunciado}

Lo probaremos para el caso discreto. Para el caso contínuo es equivalente utilizando la linealidad de la integral y los teoremas de Fubini y Tonelli. Utilizaremos el teorema de Bayes:

\begin{equation}
	\label{bayes}
	P(A|B) = \frac{P(A|B)}{P(B)}
\end{equation}

Partimos de la información mútua entre $X$ e $Y$, y llegaremos a la recíproca.

\[
	\begin{align*}
		MI(X, Y) & = -\sum_{x,y} P_{XY}(x,y) \log_2 \bigg( \frac{P_{X|Y}(x|y)}{P_X(x)} \bigg) \\
		& \stackrel{(\ref{bayes})}{=} -\sum_{x,y} P_{XY}(x,y) \log_2 \bigg( \frac{P_{XY}(x,y)}{P_X(x) P_Y(y)} \bigg) \\
		& \stackrel{(\ref{bayes})}{=} -\sum_{x,y} P_{XY}(x,y) \log_2 \bigg( \frac{P_{Y|X}(y|x)}{P_Y(y)} \bigg) = MI(Y, X)\\
	\end{align*}
\]
\qed

\begin{enunciado}
	
	Prueba la siguiente identidad entre información mútua y entropías: $MI(X, Y) = H(X) + H(Y) - H(X, Y)$.
	
\end{enunciado}

Lo probaremos para el caso discreto. Para el caso contínuo es equivalente utilizando la linealidad de la integral y los teoremas de Fubini y Tonelli. Para este y otros ejercicios utilizaremos una igualdad que conviene probar por separado. Es la siguiente, fijado un $x \in X$:

\begin{equation}
	\label{truco}
	\sum_y P_{XY}(x,y) = P_X(x)
\end{equation}

Probar esta igualdad es sencillo:

\[
	\sum_y P_{XY}(x,y) \stackrel{(\ref{bayes})}{=} \sum_y P_{Y|X}(y|x) P_X(x) = P_X(x) \underbrace{\sum_y P_{Y|X}(y|x)}_{= 1} = P_X(x)
\]

Para demostrar la igualdad pedida, partimos de la información mútua y operamos de la siguiente forma:

\[
	\begin{align*}
	MI(X, Y) & = - \sum_{x,y} P_{XY}(x,y) \log_2 \bigg( \frac{P_{XY}(x,y)}{P_X(x) P_Y(y)} \bigg) \\
	& = - \sum_{x,y} P_{XY}(x,y)  \bigg( \log_2 P_{XY}(x,y) - \log_2 P_X(x) - \log_2 P_Y(y) \bigg) \\
	& = - \sum_{x,y} P_{XY}(x,y) \log_2 P_{XY}(x,y) + \sum_{x,y} P_{XY}(x,y) \log_2 P_X(x) +\sum_{x,y} P_{XY}(x,y) \log_2 P_Y(y) \\
	& \stackrel{(\ref{truco})}{=} - \sum_{x,y} P_{XY}(x,y) \log_2 P_{XY}(x,y) + \sum_{x} P_X(x) \log_2 P_X(x) +\sum_{y} P_Y(y) \log_2 P_Y(y) \\
	& = - H(X, Y) + H(X) + H(Y)
	\end{align*}
\]
\qed


\begin{enunciado}
	
	Prueba las siguientes identidades.
	
	\begin{itemize}
		\item $H(X, Y) = H(X) + H(Y|X) = H(Y) + H(X|Y)$
		\item $MI(X, Y) = H(X) - H(X|Y) = H(Y) - H(Y|X)$
		\item $MI(X, X) = H(X)$
	\end{itemize}
	
\end{enunciado}

Probaremos estas igualdades para el caso discreto. Para el caso contínuo son equivalentes utilizando la linealidad de la integral y los teoremas de Fubini y Tonelli. Partimos de la entropía condicionada: 

\[
	\begin{align*}
		H(Y|X) & = - \sum_x P_X(x) \sum_y P_{Y|X}(y|x) \log_2 P_{Y|X}(y|x) \\
		& \stackrel{(\ref{bayes})}{=} - \sum_x P_X(x) \sum_y \frac{P_{X,Y}(x,y)}{P_X(x)} \log_2 \frac{P_{X,Y}(x,y)}{P_X(x)} \\
		& = - \sum_x \frac{P_X(x)}{P_X(x)} \sum_y P_{X,Y}(x,y) \bigg( \log_2 P_{X,Y}(x,y) - \log_2 P_X(x) \bigg) \\
		& = - \sum_{x,y} P_{X,Y}(x,y) \bigg( \log_2 P_{X,Y}(x,y) - \log_2 P_X(x) \bigg) \\
		& = - \sum_{x,y} P_{X,Y}(x,y) \log_2 P_{X,Y}(x,y) + \sum_{x,y} P_{X,Y}(x,y)\log_2 P_X(x) \\
		& \stackrel{(\ref{truco})}{=} - \sum_{x,y} P_{X,Y}(x,y) \log_2 P_{X,Y}(x,y) + \sum_x P_X(x)\log_2 P_X(x) \\
		& = H(X,Y) - H(X) 
	\end{align*}
\]

Pasamos $H(X)$ al otro miembro sumando y obtenemos la primera igualdad buscada:

\begin{equation}
	\label{entropies}
	H(X,Y) = H(X) + H(Y|X)
\end{equation}
\qed

Para probar la segunda igualdad haremos uso de la primera:

\[
	\begin{align*}
		MI(X,Y) & = H(X) + H(Y) - H(X,Y) \\
		 & \stackrel{(\ref{entropies})}{=} H(X) + H(Y) - (H(Y) + H(X|Y)) \\
		 & = H(X) - H(X|Y)
	\end{align*}
\]
\qed

Para probar la tercera igualdad necesitaremos un hacer uso de un resultado intermedio: La entropía de una variable aleatoria condicionada a sí misma es nula. Esto encaja con nuestra intuición, pues al fijar una variable aleatoria, la sorpresa asociada a la medición de la misma es trivial. Probemos este resultado intermedio:

\[
	\begin{align*}
		H(X|X) & = - \sum_{x_1} P_X(x_1) \sum_{x_2} P_{X|X}(x_2|x_1) \log_2 P_{X|X}(x_2|x_1) \\
		&  = - \sum_{x_1} P_X(x_1) P_{X|X}(x_1|x_1) \underbrace{\log_2 P_{X|X}(x_1|x_1)}_{= 0} = 0
	\end{align*}
\]

Donde hemos usado que:

\[
	P_{X|X}(x_2|x_1) =
	\begin{cases*}
		0 \quad \text{si } x_1 \neq x_2 \\
		1 \quad \text{si } x_1 = x_2
	\end{cases*}
\]

Conociendo este resultado, probar la igualdad pedida es sencillo:

\[
	\begin{align*}
		MI(X,X) & = H(X) + H(X) - H(X,X) \\
		& \stackrel{(\ref{entropies})}{=} 2 H(X) - (H(X) - \underbrace{H(X|X)}_{= 0}) = H(X)
	\end{align*}
\]
\qed

\begin{enunciado}
	
	TODO
	
\end{enunciado}



\begin{enunciado}
	
	Prueba la siguiente identidad: $H(X, Y|Z) = H(X|Z) + H(Y|X, Z)$.
	
\end{enunciado}

Lo probaremos para el caso discreto. Para el caso contínuo es equivalente utilizando la linealidad de la integral y 	los teoremas de Fubini y Tonelli. Haremos uso del teorema de Bayes para varias variables:

\begin{equation}
	\label{bayes2}
	P(A, B | C) = P(A|C) \cdot P(B|A, C)
\end{equation}

Recordemos las definiciones de entropías para varias variables que aparecen en el enunciado:

\[
\begin{align*}
	H(X, Y | Z) & = - \sum_{x,y,z} P_{XYZ}(x,y,z) \log_2 P_{XY|Z}(x, y|z) \\
	H(X|Z) & = - \sum_z P_Z(z) \sum_x P_{X|Z}(x|z) \log_2 P_{X|Z}(x|z) \\
	H(Y |X, Z) & = - \sum_{x,z} P_{XZ}(x,z) \sum_y P_{Y|XZ}(y|x,z) \log_2 P_{Y|XZ}(y|x,z)
\end{align*}
\]

Obtenemos la siguiente cadena de igualdades:

\[
	\begin{align*}
		H(X, Y | Z) & = - \sum_{x,y,z} P_{XYZ}(x,y,z) \log_2 P_{XY|Z}(x, y|z) \\
		& \stackrel{(\ref{bayes2})}{=} - \sum_{x,y,z} P_{XYZ}(x,y,z) \log_2 \bigg( P_{X|Z}(x|z) \cdot P_{Y|XZ}(y|x,z) \bigg) \\
		& = - \sum_{x,y,z} P_{XYZ}(x,y,z) \bigg( \log_2 P_{X|Z}(x|z) + \log_2 P_{Y|XZ}(y|x,z) \bigg) \\
		& = \underbrace{- \sum_{x,y,z} P_{XYZ}(x,y,z) \log_2 P_{X|Z}(x|z)}_{(A)} \underbrace{- \sum_{x,y,z} P_{XYZ}(x,y,z) \log_2 P_{Y|XZ}(y|x,z)}_{(B)} 
	\end{align*}
\]

Resta probar que los términos $(A)$ y $(B)$ son $H(X|Z)$ y $H(Y|X, Z)$ respectivamente:

\[
	\begin{align*}
		(A) & = - \sum_{x,y,z} P_{XYZ}(x,y,z) \log_2 P_{X|Z}(x|z) \\
		& = - \sum_{x,z} \log_2 P_{X|Z}(x|z) \sum_y P_{XYZ}(x,y,z)  \\
		& \stackrel{(\ref{truco})}{=} - \sum_{x,z} P_{XZ}(x,z) \log_2 P_{X|Z}(x|z) \\
		& \stackrel{(\ref{bayes})}{=} - \sum_{x,z} P_{X|Z}(x|z) P_Z(z) \log_2 P_{X|Z}(x|z) \\
		& = - \sum_z P_Z(z) \sum_x P_{X|Z}(x|z) \log_2 P_{X|Z}(x|z) = H(X|Z)
	\end{align*}
\]

\[
	\begin{align*}
		(B) & = - \sum_{x,y,z} P_{XYZ}(x,y,z) \log_2 P_{Y|XZ}(y|x,z) \\
		& \stackrel{(\ref{bayes})}{=} - \sum_{x,y,z} P_{Y|XZ}(y|x,z) P_{XZ}(x,z) \log_2 P_{Y|XZ}(y|x,z) \\
		& = - \sum_{x,z} P_{XZ}(x,z) \sum_y P_{Y|XZ}(y|x,z) \log_2 P_{Y|XZ}(y|x,z) = H(Y|X, Z)
	\end{align*}
\]
\qed

\end{document}

